\setchapterstyle{kao}
\setchapterpreamble[u]{\margintoc}
\chapter{Results and Definitions to Know}

\section{Uncategorized}

\section{Analytic Functions}

\begin{definition}[Domain]
    \labdef{domain}
    A domain is an open, connected subset of $\C$.
\end{definition}

\begin{theorem}[Cauchy Riemann Equations]
    \labthm{cauchy-riemann}
    Let $f=u + iv$ be defined on a domain $D$ in the complex plane, with both $u$ and $v$ real-valued functions. Then, $f(z)$ is analytic on $D$ if and only if $u(x,y)$ and $v(x,y)$ have continuous first-order partial derivatives that satisfy the Cauchy-Riemann equations:

    \[\frac{\partial u}{\partial x} = \frac{\partial v}{\partial y}, \quad \frac{\partial u}{\partial y} = -\frac{\partial v}{\partial x}\]

    Furthermore, the complex derivative of $f$ can be computed as
    \[f'(z) = u_x + iv_x = v_y - iu_y.\]
\end{theorem}

\begin{theorem}[Cauchy Integral Formula]
    
\end{theorem}